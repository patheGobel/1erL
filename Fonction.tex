\documentclass[12pt]{article}
\usepackage{stmaryrd}
\usepackage{graphicx}
\usepackage[utf8]{inputenc}

\usepackage[french]{babel}
\usepackage[T1]{fontenc}
\usepackage{hyperref}
\usepackage{verbatim}

\usepackage{color, soul}

\usepackage{pgfplots}
\pgfplotsset{compat=1.15}
\usepackage{mathrsfs}

\usepackage{amsmath}
\usepackage{amsfonts}
\usepackage{amssymb}
\usepackage{tkz-tab}
\author{Destinés à la 1\textsuperscript{ère}L\\Au Lycée de Dindéferlo}
\title{\textbf{GENERALITES SUR LES FONCTIONS NUMERIQUES}}
\date{\today}
\usepackage{tikz}
\usetikzlibrary{arrows, shapes.geometric, fit}

% Commande pour la couleur d'accentuation
\newcommand{\myul}[2][black]{\setulcolor{#1}\ul{#2}\setulcolor{black}}
\newcommand\tab[1][1cm]{\hspace*{#1}}

\begin{document}
\maketitle
\newpage
\section*{\underline{\textbf{\textcolor{red}{I.Fonction numérique d’une variable réelle}}}}
\subsection*{\underline{\textbf{\textcolor{red}{1.Définition et Notation}}}}
\subsection*{\underline{\textbf{\textcolor{red}{Exemple}}}}
\subsection*{\underline{\textbf{\textcolor{red}{Solution}}}}
\subsection*{\underline{\textbf{\textcolor{red}{3.Ensemble de définition d’une fonction}}}}
\subsection*{\underline{\textbf{\textcolor{red}{a.Définition}}}}
\subsection*{\underline{\textbf{\textcolor{red}{b.Ensemble de définition d’une fonction polynôme}}}}
\subsection*{\underline{\textbf{\textcolor{red}{c.Fraction rationnelle}}}}
\subsection*{\underline{\textbf{\textcolor{red}{Exemple}}}}
\section*{\underline{\textbf{\textcolor{red}{II.Parité d’une fonction}}}}
\subsection*{\underline{\textbf{\textcolor{red}{1.Ensemble symétrique par rapport à zéro}}}}
\subsection*{\underline{\textbf{\textcolor{red}{a.Définition}}}}
\subsection*{\underline{\textbf{\textcolor{red}{b.Exemples et contre-exemples}}}}
\subsection*{\underline{\textbf{\textcolor{red}{2.Fonction paire et fonction impaire}}}}
\subsection*{\underline{\textbf{\textcolor{red}{a.Fonction paire}}}}
\subsection*{\underline{\textbf{\textcolor{red}{b.Fonction impaire}}}}
\subsection*{\underline{\textbf{\textcolor{red}{c.Remarque}}}}
$\bullet$ Si l’ensemble de définition $D_{f}$ de f n’est pas symétrique par rapport à zéro ou bien si \( f(-x) \neq f(x)\) et \(f(-x) \neq f(-x) \) alors f n’est ni une fonction paire ni une fonction impaire.\\
$\bullet$ Etudier la parité d’une fonction f, c’est étudier si la fonction f est paire ou bien impaire.
\subsection*{\underline{\textbf{\textcolor{red}{d.Exercice d’application}}}}
Etudier la parité des fonctions définies ci-dessous :\\
\(f(x)=\frac{x^{4}}{x^{2}-4}\)\\
\(g(x)=\frac{x^{3}}{x^{2}+1}\)\\
\(h(x)=\frac{3x-1}{x-3}\)\\
\end{document}