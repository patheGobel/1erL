\documentclass[12pt]{article}
\usepackage{stmaryrd}
\usepackage{graphicx}
\usepackage[utf8]{inputenc}

\usepackage[french]{babel}
\usepackage[T1]{fontenc}
\usepackage{hyperref}
\usepackage{verbatim}

\usepackage{color,soul}

\usepackage{amsmath}
\usepackage{amsfonts}
\usepackage{amssymb}
\usepackage{tkz-tab}

\usepackage{tabularray}
\usepackage{ninecolors}

\author{Destinés à la 1erL\\Au Lycée de Dindéferlo}
\title{\textbf{Polynômes}}
\date{\today}
\usepackage{tikz}
\usetikzlibrary{arrows}
%This command takes a colour as an optional argument; the default colour is black.
\usetikzlibrary{shapes.geometric,fit}
\newcommand{\myul}[2][black]{\setulcolor{#1}\ul{#2}\setulcolor{black}}
\newcommand\tab[1][1cm]{\hspace*{#1}}
\begin{document}
\maketitle
\newpage


\underline{\textbf{\textcolor{red}{I°)Généralités}}}\\
\underline{\textbf{\textcolor{red}{1.Mônomes}}}\\
\underline{\textbf{\textcolor{red}{a.Définition et vocabulaire}}}\\
On appelle monôme de la variable $x$, toute expression qui peut s'écrire sous la forme $ax^{n}$où $a$ est un réel constant et $n$ est un entier naturel.\\
Le mônome a trois parties:\\
$\bullet$Le réel a est appelé coefficient du monôme $ax^{n}$.\\
$\bullet$x est la variable du monôme $ax^{n}$.\\
$\bullet$n est un entier naturel appelé degré du monôme $ax^{n}$.\\
Si le coefficient a est différent de 0 alors on dit que le monôme $ax^{n}$ est de degré $n$ et on note $deg(ax^{n})=n$ \\
\underline{\textbf{\textcolor{red}{b.Exemples de monôme}}}\\
$\ast$ L'expression $-3x^{7}$ est un monôme de la variable x de coefficient -3 et de degré 7.\\
$\ast$ L'expression $\frac{2}{5}x^{2}$ est un monôme de la variable x de coefficient $\frac{2}{5}$ et de degré 2.\\
$\ast$ L'expression $x^{3} \sqrt{7} $ est un monôme de la variable x de coefficient $\sqrt{7}$ et de degré 3.\\
$\ast$ L'expression $\frac{1}{7}x$ est un monôme de la variable x de coefficient $\frac{1}{7}$  et de degré 1.\\
\underline{\textbf{\textcolor{red}{Attention}}}\\L'expression $9x^{-3}$ n'est pas un monôme car l'exposant -3 n'est pas un entier naturel.\\
%\underline{\textbf{\textcolor{red}{c.Remarque}}}\\

\underline{\textbf{\textcolor{red}{2.Polynômes}}}\\
\underline{\textbf{\textcolor{red}{a.Définition}}}\\
On appelle polynôme de la variable x, toute expression qui peut s'écrire comme somme de monômes de la variable x.\\
Un polynôme se note généralement par $P(x)$ ou $Q(x)$ ...\\
\underline{\textbf{\textcolor{red}{b.Degré d'un polynôme}}}\\
Le degré d'un polynôme est le degré du monôme le plus haut.\\ 
\underline{\textbf{\textcolor{red}{c.Exemple}}}\\
$\circ$L'expression $P(x)=4x^{3}-2x+3$ est un polynôme car il s'écrit comme la somme des monômes $4x^{3}$ ;$-2x$ et 3\\ %On note $P(x)$=$4x^{3}-2x+3$\\
$\bullet$Les réels -2 ; 4 et 1 sont dits coefficients.\\
$\bullet$ 3 est le plus grand parmi tous les degrés des monômes du polynôme $4x^{3}-2x+3$\\
donc le degré du polynôme $4x^{3}-2x+3$ est 3.\\
$\circ$ L'expression $P(x)=-2x+4x^{2}+3$ est un polynôme de $deg(P)=3$\\
%\underline{\textbf{\textcolor{red}{c.Remarque}}}\\
\underline{\textbf{\textcolor{red}{II°)Trinômes du second degré}}}\\
\underline{\textbf{\textcolor{red}{1.Définition}}}\\
Un trinôme du $2^{nd}$ degré est une expression qui peut s'écrire sous la forme $ax^{2}+bx+c$ où $a$ est un réel non nul, $b$ et $c$ sont deux réels quelconques.C'est donc un polynôme de degré 2.\\ 
\underline{\textbf{\textcolor{red}{1.Exemple}}}\\
$5x^{2}-2x+1$ est un trinôme du second degré avec $a=5$ ; $b=-2$ et $c=1$.\\
\underline{\textbf{\textcolor{red}{2.Factorisation de $ax^{2}+bx+c$}}}\\
Soit $P(x)=ax^{2}+bx+c$.\\
Le réel $b^{2}-4ac$ est dit discriminant de $ax^{2}+bx+c$ et il est noté $\Delta=b^{2}-4ac$.\\ 
La factorisation de $ax^{2}+bx+c$ dépend de $\Delta$.

\begin{tabular}{|c|c|}
    \hline
    Signe du discriminant $\Delta$ & Factorisation du trinôme $ax^{2}+bx+c$ \\
    \hline
    $\Delta<0$ & $ax^{2}+bx+c$ ne peut pas se factoriser \\
    \hline
    $\Delta=0$) & $ax^{2}+bx+c=a(x-x_{0	})^{2} où x_{0}=\frac{-b}{2a}$ \\
    \hline
    $\Delta>0$ & $ax^{2}+bx+c=a(x-x_{1})(x-x_{2})$ où $x_{1}=\frac{-b+\sqrt{\Delta}}{2a}$ et $x_{1}=\frac{-b-\sqrt{\Delta}}{2a}$ \\
    \hline
\end{tabular}

\underline{\textbf{\textcolor{red}{Exemples:}}}\\
\underline{\textbf{\textcolor{red}{3.Signe de $ax^{2}+bx+c$}}}\\
Le signe de $ax^{2}+bx+c$ s'obtient généralement à l'aide d'un tableau de signes et ce tableau dépend du signe de $\Delta$.\\
$\bullet$ Si $\Delta<0$ alors $ax^{2}+bx+c$ n'a pas de racine et son tableau de signes est le suivant :\\
$\bullet$ Si $\Delta=0$ alors $ax^{2}+bx+c$ a une racine double $\frac{-b}{2a}$ et son tableau de signes est le suivant :\\
\underline{\textbf{\textcolor{red}{Exemple:}}}\\
\underline{\textbf{\textcolor{red}{Exemple:}}}\\
\underline{\textbf{\textcolor{red}{III.Théorème fondamental des polynômes}}}\\
\underline{\textbf{\textcolor{red}{1.Racine d'un polynôme}}}\\
\underline{\textbf{\textcolor{red}{a.Définition}}}\\
Soit $P(x)$ un polynôme et $\alpha$ un réel. 
On dit que $\alpha$ est une racine (ou zéro) de $P(x)$ si $P(\alpha)=0$\\
\underline{\textbf{\textcolor{red}{b.Exemple:}}}\\
$P(x)=x^{3}-2x^{2}-5x+6$\\
Vérifions que 1 est une racine de $P(x)$. c'est-à-dire $P(1)=0$\\
$P(1)=1^{3}-2(1^{2})-5(1)+6=0$ on a $P(1)=0$ \\Donc 1 est une racine de $P(x)$\\
\underline{\textbf{\textcolor{red}{c.Exercice d'application}}}\\
Vérifier que -2 est une racine de $P(x)=x^{3}+6x^{2}+11x+6$\\
\underline{\textbf{\textcolor{red}{b.Théorème}}}\\
Si $\alpha$ est une racine d'un polynôme $P(x)$ alors $P(x)$ est factorisable par $(x-\alpha)$.\\
Ainsi il existe un polynome $Q(x)$ tel que $P(x)=(x-\alpha)Q(x)$ avec\\ $deg(Q)<deg(P)$\\
\underline{\textbf{\textcolor{red}{Exemple:}}}\\
En considérant l'exemple précédent on a\\
$P(x)=x^{3}-2x^{2}-5x+6$ Puisque $P(1)=0$ donc $P(x)$ est factorisable par $(x-1)$\\
Donc $P(x)=(x-1)Q(x)$ \\
\underline{\textbf{\textcolor{red}{NB:}}}\\
Dans la suite du cours l'objectif sera de trouver l'expression de $Q(x)$\\
\underline{\textbf{\textcolor{red}{3. Détermination du polynôme $Q(x)$ par la division euclidienne}}}\\
Si un polnôme $P(x)$ est factorisable par $(x-\alpha)$ alors $P(x)$ divisé par $(x-\alpha)$ donne un reste égale à zéro.\\
C'est-à-dire [Pose La division]\\

\underline{\textbf{\textcolor{red}{a.Exemple}}}\\
    \begin{tblr}{
            colspec={rrrrrrrr|rrrrr},
            cells={mode=dmath},
            colsep = .5mm } 
        \SetCell[r=2]{fg=red}- 
        & 6x^3 & - & 2x^2 & + & x & + & 3 & x^2 & - & x & + & 1 \\
\SetHline{9-13}{blue,0.5pt}
& 6x^3 & - & 6x^2 & + & 6x &  &  &  &  & 6x & + & 4 \\
\SetHline{2-8}{blue,0.5pt}
        %or \cline{2-8}
        \SetCell[r=2,c=3]{fg=red}-
&  &  & 4x^2 & - & 5x & + & 3 &  &  &  &  &  \\
&  &  & 4x^2 & - & 4x & + & 4 &  &  &  &  &  \\
        \SetHline{4-8}{blue,0.5pt}
        %\cline{4-8}
        &  &  &  & - & x & - & 1 &  &  &  &  &  \\
    \end{tblr}
    \\
\underline{\textbf{\textcolor{red}{b.Exercice d'application}}}\\
\underline{\textbf{\textcolor{red}{3. Détermination du polynôme $Q(x)$ par la méthode d'identification des coefficients}}}\\
\underline{\textbf{\textcolor{red}{b.Exercice d'application}}}\\
\underline{\textbf{\textcolor{red}{a.Exemple}}}\\
\underline{\textbf{\textcolor{red}{3. Détermination du polynôme $Q(x)$ par la méthode de Horner}}}\\
\underline{\textbf{\textcolor{red}{b.Exercice d'application}}}\\
\underline{\textbf{\textcolor{red}{a.Exemple}}}\\


\underline{\textbf{\textcolor{red}{III) Factorisation d’un polynôme de degré n :}}}\\

\underline{\textbf{\textcolor{red}{IV) Equations et Inéquations}}}\\

\end{document}