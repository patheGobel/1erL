\documentclass[12pt]{article}
\usepackage{stmaryrd}
\usepackage{graphicx}
\usepackage[utf8]{inputenc}

\usepackage[french]{babel}
\usepackage[T1]{fontenc}
\usepackage{hyperref}
\usepackage{verbatim}

\usepackage{color,soul}

\usepackage{amsmath}
\usepackage{amsfonts}
\usepackage{amssymb}
\usepackage{tkz-tab}
\usepackage{envmath}
\author{Destinés à la 1erL\\Au Lycée de Dindéferlo}
\title{\textbf{Systèmes D'Equations et D'Inequations linéaires à trois inconnues}}
\date{\today}
\usepackage{tikz}
\usetikzlibrary{arrows}
%This command takes a colour as an optional argument; the default colour is black.
\usetikzlibrary{shapes.geometric,fit}
\newcommand{\myul}[2][black]{\setulcolor{#1}\ul{#2}\setulcolor{black}}
\newcommand\tab[1][1cm]{\hspace*{#1}}
\begin{document}
\maketitle
\newpage


\underline{\textbf{\textcolor{red}{I°)Systèmes D'Equations}}}\\


\underline{\textbf{\textcolor{red}{1.Système de deux équations linéaires à 2 inconnues}}}\\

\underline{\textbf{\textcolor{red}{2.Systèmes de deux équations linéaires à 3 ou 4 inconnues (pivot de Gauss)}}}\\


\underline{\textbf{\textcolor{red}{II°)Systèmes D'inequations}}}\\

\underline{\textbf{\textcolor{red}{1.Systèmes de deux inequations linéaires à 2 inconnues}}}\\

\underline{\textbf{Exemple}}\\

\begin{System}
  x -2y+1\geq0 \\
  2x + y -3<0
\end{System}
est un système de deux inéquations linéaires à deux inconnues x et y\\

\textbf{Résolution par la méthode graphique}\\
On commence par représenter graphiquement la droite $(D_{1})$ d'équation \\$x -2y+1=0$ et la droite $(D_{2})$ d'équation $2x + y -3=0$ dans un repère \\orthonormé\\

Puis on choisit un point qui n'est ni sur $(D_{1})$, ni sur $(D_{2})$ et dont les coordonnées sont connues.\\
Par exemple le point O(0,0).\\
\\
En remplaçant $x$ et $y$ par les coordonnées de O dans l'inéquation 2, on a : 
$0-2(0)+1\geq0$ c'est à dire $1\geq0$ vrai donc les cordonnées de O vérifie l'inéquation 1.
\\
On barre donc le demi-plan de frontière $(D_{1})$ ne contenant pas O.\\
\\
En remplaçant $x$ et $y$ par les coordonnées de O dans l'inéquation 2, on a : 
$2(0)+0-3<0$ c'est à dire $-3<0$ vrai donc les cordonnées de O vérifie l'inéquation 2.
\\
On barre donc le demi-plan de frontière $(D_{2})$ ne contenant pas O.\\
\underline{\textbf{\textcolor{red}{1.Systèmes de deux inéquations linéaires à 3 inconnues.}}}\\
\\
\underline{\textbf{Exemple}}\\

\begin{System}
  -x+y+2\geq0 \\
   2x-y-4<0 \\
   x - 2y -1>0
\end{System}
est un système de trois inéquations à deux inconnues x et y.\\
\textbf{Résolution par la méthode graphique}\\
\\
\underline{\textbf{\textcolor{red}{2.Problèmes d'optimisation}}}\\

\end{document}