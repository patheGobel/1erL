\documentclass[12pt]{article}
\usepackage{stmaryrd}
\usepackage{graphicx}
\usepackage[utf8]{inputenc}

\usepackage[french]{babel}
\usepackage[T1]{fontenc}
\usepackage{hyperref}
\usepackage{verbatim}

\usepackage{color, soul}

\usepackage{pgfplots}
\pgfplotsset{compat=1.15}
\usepackage{mathrsfs}

\usepackage{amsmath}
\usepackage{amsfonts}
\usepackage{amssymb}
\usepackage{tkz-tab}
\author{Destinés à la 1\textsuperscript{ère}L\\Au Lycée de Dindéfelo}
\title{\textbf{Chapitre 5 : Dérivabilité\\Taisez-vous et suivez!!!}}
\date{\today}
\usepackage{tikz}
\usetikzlibrary{arrows, shapes.geometric, fit}

% Commande pour la couleur d'accentuation
\newcommand{\myul}[2][black]{\setulcolor{#1}\ul{#2}\setulcolor{black}}
\newcommand\tab[1][1cm]{\hspace*{#1}}

\begin{document}
\maketitle
\newpage
\section*{\underline{\textbf{\textcolor{red}{I.Dérivabilité d’une fonction en un nombre réel}}}}
\subsection*{\underline{\textbf{\textcolor{red}{1. Définition}}}}
On dit qu’une fonction f est dérivable en un réel $\alpha$ $(\alpha \in D_{f})$ si \[\lim_{x \to \alpha}\frac{f(x)-f(\alpha)}{x-\alpha}=\mathit{l}\] où $\mathit{l}$ est un nombre réel.\\
Le nombre réel l est appelé nombre dérivé de f en $\alpha$ et est noté $f'(x)$
\subsubsection*{\textcolor{red}{Exemple 1}}
$f(x)=2x-5$ ; Montrons que f est dérivable en 1 et précisons le nombre dérivé de f en 1.
\subsubsection*{\textcolor{red}{Solution 1}}
\[\lim_{x \to 1}\frac{f(x)-f(1)}{x-1}=\lim_{x \to 1}\frac{2x-5-(2(1)-5)}{x-1}=\lim_{x \to 1}\frac{2x-2}{x-1}=\lim_{x \to 1}\frac{2(x-1)}{x-1}=2\]\\
Donc \[\lim_{x \to 1}\frac{f(x)-f(1)}{x-1}=2\]donc f est dérivable en 1 et le nombre dérivé de f en 1 est f'(1)=2.
\subsubsection*{\textcolor{red}{3. Exercice d’application}}
Soit $f(x)=-7x+2$. Montrer que f est dérivable en 3 et préciser le nombre dérivé de f en 3.
\subsubsection*{\textcolor{red}{4. Tangente à la courbe d’une fonction en un point}}
\subsection*{\underline{\textbf{\textcolor{red}{a. Définition}}}}
Soit f est une fonction dérivable en a. La droite d’équation\\ $y=f'(x)(x-a)+f(a)$ est dite tangente à la courbe de f au point d’abscisse a.
\subsubsection*{\textcolor{red}{b. Exemple}}
Soit f telle que $f(x)= x^{2}$ . On montre que f est dérivable en 1 et le nombre dérivé de f en 1 est
$f'(1)=2$. Ainsi la tangente à la courbe de f au point d’abscisse 1 est la droite d’équation $y =f'(1)(x-1)+f(1) = 2x-1.$
\section*{\textcolor{red}{II.Fonction dérivée d’une fonction donnée}}
Soit f une fonction dérivable en tout nombre réel a d’un intervalle I. A partir de la fonction, on peut définir une nouvelle fonction notée f' appelée fonction dérivée de f. L’expression de la fonction f' est donc f'(x).
\subsubsection*{\textcolor{red}{1. Fonction dérivée des fonctions usuelles}}
Les propriétés suivantes permettent de calculer les expressions des fonctions dérivées des
fonctions usuelles.
\subsubsection*{\textcolor{red}{a. Propriété}}
Si f(x) = c où c est un réel constant alors la fonction dérivée de f est définie par f'(x) = 0.
\subsubsection*{\textcolor{red}{Exemples}}
\begin{itemize}
\item Pour $f(x) = 8$, on a $f'(x) = 0$
\item Pour $f(x)=-5$, on a $f'(x) = 0$
\end{itemize}
\subsubsection*{\textcolor{red}{b. Propriété}}
Si f(x)=x alors f'(x)= 1.
\subsubsection*{\textcolor{red}{c. Propriété}}
Si $f(x) = ax+b$ où a et b sont des réels constants alors $f'(x)=a$
\subsubsection*{\textcolor{red}{Exemples}}
\begin{itemize}
\item Soit $f(x)=2x$, ici $a=2$ et $b=0$ donc $f'(x)=2$.
\item Soit $f(x)=-x+4$, $a=-1$ et $b=4$ donc $f'(x) = -1$.
\end{itemize}
\subsubsection*{\textcolor{red}{d. Propriété}}
Si $f(x)=x^{n}$ où n est un entier naturel non nul alors $f'(x)=nx^{n-1}$
\subsubsection*{\textcolor{red}{Exemples}}
\begin{itemize}
\item Soit $f(x) = x^{2}$ ; $f(x) = x^{n}$ avec $n=2$ donc $f'(x) = 2x$.
\item Soit $f(x) = x^{3}$ ; $f(x) = x^{n}$ avec $n=3$ donc $f'(x) = 3x$.
\end{itemize}
\subsubsection*{\textcolor{red}{e. Propriété}}
Si $f(x)=\frac{1}{x}$ alors $f'(x)=-\frac{1}{x^{2}}$
\subsubsection*{\textcolor{red}{Tableau récapitulatif}}
Le tableau suivant permet de résumer les résultats ci-dessus\\
\begin{tabular}{|c|c|}
\hline
$f(x)$ & $f'(x)$ \\
\hline
$f(x)=c$; $c\in \mathbb{R}$ & $f'(x)=0$ \\
\hline
$f(x)=x$ & $f'(x)=1$\\
\hline
$f(x)=ax+b$ & $f'(x)=a$\\
\hline
$f(x)=x^{n}$;$n \in \mathbb{N}\setminus\left\lbrace 0\right\rbrace $ & $f'(x)=nx^{n-1}$\\
\hline
$f(x)=\frac{1}{x}$ & $f'(x)=-\frac{1}{x^{2}}$\\
\hline
\end{tabular}
\subsubsection*{\textcolor{red}{2. Opérations sur les dérivées}}
\subsubsection*{\textcolor{red}{a. Dérivée d’une somme}}
\begin{itemize}
\item Soient $u$ et $v$ deux fonctions. La dérivée de la somme $u(x) + v(x)$ est\\
\textcolor{red}{$[u(x) + v(x)]'=u'(x) + v'(x)$}
\end{itemize}
\subsection*{\textcolor{red}{Exemple}}
Pour $f(x)=x^{4}+x^{3}$ on a $f'(x)=4x^{3}+3x^{2}$
\begin{itemize}
\item Soient $u$ et $v$ deux fonctions. La dérivée de la somme $u(x) + v(x)$ est\\
\textcolor{red}{$[u(x) + v(x)]'=u'(x) - v'(x)$}
\end{itemize}
\subsubsection*{\textcolor{red}{Exemple}}
Pour $f(x)=3x-x^{2}$ on a $f'(x)=3-2x$
\subsubsection*{\textcolor{red}{b. Dérivée d’un produit}}
\begin{itemize}
\item Soit $u$ une fonction et $\alpha$ un nombre réel constant non nul. La dérivée du produit 
$\alpha u(x)$
est : \textcolor{red}{$[\alpha u(x)]' = \alpha u'(x)$}
\end{itemize}
\subsubsection*{\textcolor{red}{Exemples}}
$f(x) =4x^{3}$ , on a : $f'(x) = 4(3x^{2}) =12x^{2} $
\begin{itemize}
\item Soient $u$ et $v$ deux fonctions. La dérivée du produit $u(x) \times v(x)$ est : 
\textcolor{red}{$[u(x) \times v(x)]' =u'(x) \times v(x) + v'(x) \times u(x)$}
\end{itemize}
\subsubsection*{\textcolor{red}{Exemple}}
Pour $f(x)=(3x+1)(x^{3}+x)$ . On a $f'(x)=3(x^{3}+x)+3x^{2}(3x+1)$.
\begin{itemize}
\item Soit $u$ une fonction et $n$ un entier naturel supérieur à $1$.\\
La dérivée de $[u(x)]^{n}$ est :\textcolor{red}{$[u(x)n]' = n \times u'(x) \times [u(x)]^{n-1}$}
\end{itemize}
\subsubsection*{\textcolor{red}{Exemple}}
Pour $f(x) =(-2x + 5)^{3}$ , on a $f'(x) = 3(-2)(-2x + 5)^{2} =-6(-2x + 5)^{2}$
\subsubsection*{\textcolor{red}{c. Dérivée d’un quotient}}
\begin{itemize}
\item Soit u une fonction. La dérivée du quotient $\frac{1}{u(x)}$ est  
$\left[ \frac{1}{u(x)}\right]'=-\frac{u'(x)}{\left[ u(x)\right]^{2}}$
\end{itemize}
\subsubsection*{\textcolor{red}{Exemples}}
Pour $f(x) =\frac{1}{x+1}$ on a :$f(x)=-\frac{1}{(2x+1)^{2}}$
\begin{itemize}
\item Soient u et v deux fonctions dérivables sur un intervalle I tel que v ne s’annule pas sur I. La dérivée du quotient  $\frac{u(x)}{v(x)}$ est :\\
\textcolor{red}{$\left[\frac{u(x)}{v(x)}\right]'=\frac{u'(x)v(x)-v'(x)u(x)}{\left[ v(x)\right] ^{2}}$}
\end{itemize}
\subsubsection*{\textcolor{red}{Exemple}}
Pour $f(x)=\frac{2x-1}{3x+1}$, on a 
$f'(x)=\frac{2(3x+1)-3(2x-1)}{(3x+1)^{2}}$=$\frac{5}{(3x+1)^{2}}$\\
Le tableau ci-dessous permet de résumer les différents résultats ci-dessus\\
\begin{tabular}{|c|c|}
\hline
Fonctions définies par & Dérivées\\
\hline
$u(x)+v(x)$ & $u'(x)+v'(x)$ \\
\hline
$u(x)-v(x)$&$u'(x)-v'(x)$  \\
\hline
$\alpha\times u(x)$ & $\alpha\times u'(x)$\\
\hline
$u(x)\times v(x)$ & $u'(x) \times v(x) + v'(x) \times u(x)$\\
\hline
$\frac{1}{u(x)}$ & $-\frac{u'(x)}{\left[ u(x)\right] ^{2}}$\\
\hline
$\frac{u(x)}{v(x)}$ & $\frac{u'(x)v(x)-v'(x)u(x)}{\left[ v(x)\right] ^{2}}$\\
\hline
$[u(x)]^{n}$&$n \times u'(x) \times [u(x)]^{n-1}$\\
\hline
\end{tabular}
\subsection*{\textcolor{red}{III.Sens de variation d’une fonction}}
\subsubsection*{\textcolor{red}{1. Théorème}}
Soit f est une fonction dérivable en tout nombre réel a d’un intervalle I.\\
\begin{itemize}
\item Si pour tout $x \in I, f'(x)\geq 0$ alors on dit que f est croissante sur I.
\item Si pour tout $x \in I, f'(x)\leq 0$ alors on dit que f est décroissante sur I.
\item Si pour tout $x \in I, f'(x)=0$ alors on dit que f est constante sur I.
\end{itemize}
\subsubsection*{\textcolor{red}{2. Définition}}
Etudier le sens de variation d’une fonction f sur un intervalle I, c’est étudier si f est croissante ou décroissante sur I.\\\\
Ainsi pour étudier le sens de variation (ou les variations) d’une fonction sur un intervalle I alors on calcule f'(x) puis on étudie son signe sur I.\\
\subsubsection*{\textcolor{red}{3. Exemple}}
$f(x)=x^{2}-3x+2$. Etudions le sens de variation de f sur les intervalles de $D_{f}$.\\
Dressons le tableau qui permet de visualiser les variations de f, ce tableau sera appelé tableau de variations de f.\\
En effet,\\
$f'(x)=2x-3$ Cherchons le signe de f'.\\
Posons $f'(x)=0 \Longrightarrow 2x-3=0 \Longrightarrow x=\frac{3}{2}$\\
\definecolor{cqcqcq}{rgb}{0.7529411764705882,0.7529411764705882,0.7529411764705882}
\begin{tikzpicture}[line cap=round,line join=round,>=triangle 45,x=1cm,y=1cm]
%\draw [color=cqcqcq,, xstep=1cm,ystep=1cm] (-7,-10) grid (-22,17);
\clip(-22,-5) rectangle (12,10);
\draw [line width=2pt] (-23,8)-- (-7,8); %première ligne A(-22,8)---B(-7,8)
\draw [line width=2pt] (-22,6)-- (-7,6); %deuxième ligne
\draw [line width=2pt] (-22,4)-- (-7,4); %troisième ligne
\draw [line width=2pt] (-22,4)-- (-22,8); %première colonne (x=-22,y=4)"cordone de en bas
\draw [line width=2pt] (-19,8)-- (-19,4); %deuxième colone
\draw [line width=2pt] (-7,8)-- (-7,4); %troisième colonne
\draw (-21,5.5) node[anchor=north west] {$f'(x)=2x-3$};
\draw (-21,7) node[anchor=north west] {$x$};
\draw (-19,7) node[anchor=north west] {$-\infty$};
\draw (-8,7) node[anchor=north west] {$+\infty$};
%signe de la dérivé
\draw (-15.5,5.3) node[anchor=north west] {$-$};
\draw (-13.3,5.3) node[anchor=north west] {$O$};
\draw (-10.5,5.3) node[anchor=north west] {$+$};
%zéro de ln
\draw [line width=2pt] (-13,6)-- (-13,4);
\draw (-13.2,6.9) node[anchor=north west] {$\frac{3}{2}$};
\end{tikzpicture}
D'après le tableau,\\
si $x \in \left] -\infty;\frac{3}{2} \right[ $ alors $f'(x)<0$ donc f est décroissante sur 
$x \in \left] -\infty;\frac{3}{2} \right[$\\
si $x \in \left]  \frac{3}{2};+\infty \right[$ alors $f'(x)>0$ donc f est croissante sur
$x \in \left]  \frac{3}{2};+\infty \right[$\\\\ 
$f(x)=x^{3}-3x$.Etudions le sens de variation de f sur les intervalles de $D_{f}$
puis dressons son tableau de variations.\\
Dressons le tableau qui permet de visualiser les variations de f, ce tableau sera appelé tableau de variations de f.\\
En effet,\\
\subsubsection*{\textcolor{red}{4. Extrémums d’une fonction}}
Si f'(x) s’annule en a et change de signe alors f admet un extrémum en a et dans ce cas,
l’extrémum est le point de coordonnée (a ;f(a)). De plus si le signe de f'(x) passe de - en $-$alors l’extrémum est dit maximum et si c’est de $-$ en + alors il est dit minimum.
Par exemple f définie ci-dessus admet un extrémum en 3 et cet extrémum est un minimum de
f.
\end{document}